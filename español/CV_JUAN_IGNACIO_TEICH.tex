% vim: set textwidth=120:

% Example CV based on the 1.5-column-cv template. Main features:
% * uses the Roboto font family and IcoMoon icon set;
% * doesn't use colours, different font weights are used instead for styling;
% * because the CV fits on one page, header and footer is empty, since there isn't much useful info to put there;
% * includes a photo.
\documentclass[a4paper,10pt]{article}


% package imports
% ---------------

\usepackage[spanish]{babel} % for correct language and hyphenation and stuff
\usepackage{calc}           % for easier length calculations (infix notation)
\usepackage{enumitem}       % for configuring list environments
\usepackage{fancyhdr}       % for setting header and footer
\usepackage{fontspec}       % for fonts
\usepackage{geometry}       % for setting margins (\newgeometry)
\usepackage{graphicx}       % for pictures
\usepackage{microtype}      % for microtypography stuff
\usepackage{xcolor}         % for colours
\usepackage{outlines}
\usepackage{roboto}
% margin and column widths
% ------------------------

% margins
\newgeometry{left=15mm,right=15mm,top=15mm,bottom=15mm}

% width of the gap between left and right column
\newlength{\cvcolumngapwidth}
\setlength{\cvcolumngapwidth}{3.5mm}

% left column width
\newlength{\cvleftcolumnwidth}
\setlength{\cvleftcolumnwidth}{44.5mm}

% right column width
\newlength{\cvrightcolumnwidth}
\setlength{\cvrightcolumnwidth}{\textwidth-\cvleftcolumnwidth-\cvcolumngapwidth}

% set paragraph indentation to 0, because it screws up the whole layout otherwise
\setlength{\parindent}{0mm}


% style definitions
% -----------------
% style categories explanation:
% * \cvnameXXX is used for the name;
% * \cvsectionXXX is used for section names (left column, accompanied by a horizontal rule);
% * \cvtitleXXX is used for job/education titles (right column);
% * \cvdurationXXX is used for job/education durations (left column);
% * \cvheadingXXX is used for headings (left column);
% * \cvmainXXX (and \setmainfont) is used for main text;
% * \cvruleXXX is used for the horizontal rules denoting sections.

% font families
\defaultfontfeatures{Ligatures=TeX} % reportedly a good idea, see https://tex.stackexchange.com/a/37251

\newfontfamily{\cvnamefont}{Roboto Medium}
\newfontfamily{\cvsectionfont}{Roboto Medium}
\newfontfamily{\cvtitlefont}{Roboto Regular}
\newfontfamily{\cvdurationfont}{Roboto Light Italic}
\newfontfamily{\cvheadingfont}{Roboto Regular}
\setmainfont{Roboto Light}

% colours
\definecolor{cvnamecolor}{HTML}{000000}
\definecolor{cvsectioncolor}{HTML}{000000}
\definecolor{cvtitlecolor}{HTML}{000000}
\definecolor{cvdurationcolor}{HTML}{000000}
\definecolor{cvheadingcolor}{HTML}{000000}
\definecolor{cvmaincolor}{HTML}{000000}
\definecolor{cvrulecolor}{HTML}{000000}

\color{cvmaincolor}

% styles
\newcommand{\cvnamestyle}[1]{{\Large\cvnamefont\textcolor{cvnamecolor}{#1}}}
\newcommand{\cvsectionstyle}[1]{{\normalsize\cvsectionfont\textcolor{cvsectioncolor}{#1}}}
\newcommand{\cvtitlestyle}[1]{{\large\cvtitlefont\textcolor{cvtitlecolor}{#1}}}
\newcommand{\cvdurationstyle}[1]{{\small\cvdurationfont\textcolor{cvdurationcolor}{#1}}}
\newcommand{\cvheadingstyle}[1]{{\normalsize\cvheadingfont\textcolor{cvheadingcolor}{#1}}}


% inter-item spacing
% ------------------

% vertical space after personal info and standard CV items
\newlength{\cvafteritemskipamount}
\setlength{\cvafteritemskipamount}{5mm plus 1.25mm minus 1.25mm}

% vertical space after sections
\newlength{\cvaftersectionskipamount}
\setlength{\cvaftersectionskipamount}{2mm plus 0.5mm minus 0.5mm}

% extra vertical space to be used when a section starts with an item with a heading (e.g. in the skills section),
% so that the heading does not follow the section name too closely
\newlength{\cvbetweensectionandheadingextraskipamount}
\setlength{\cvbetweensectionandheadingextraskipamount}{1mm plus 0.25mm minus 0.25mm}


% intra-item spacing
% ------------------

% vertical space after name
\newlength{\cvafternameskipamount}
\setlength{\cvafternameskipamount}{3mm plus 0.75mm minus 0.75mm}

% vertical space after personal info lines
\newlength{\cvafterpersonalinfolineskipamount}
\setlength{\cvafterpersonalinfolineskipamount}{2mm plus 0.5mm minus 0.5mm}

% vertical space after titles
\newlength{\cvaftertitleskipamount}
\setlength{\cvaftertitleskipamount}{1mm plus 0.25mm minus 0.25mm}

% value to be used as parskip in right column of CV items and itemsep in lists (same for both, for consistency)
\newlength{\cvparskip}
\setlength{\cvparskip}{0.5mm plus 0.125mm minus 0.125mm}

% set global list configuration (use parskip as itemsep, and no separation otherwise)
\setlist{parsep=0mm,topsep=0mm,partopsep=0mm,itemsep=\cvparskip}


% CV commands
% -----------

% creates a "personal info" CV item with the given left and right column contents, with appropriate vertical space after
% @param #1 left column content (should be the CV photo)
% @param #2 right column content (should be the name and personal info)
\newcommand{\cvpersonalinfo}[2]{
    % left and right column
    \begin{minipage}[t]{\cvleftcolumnwidth}
        \vspace{0mm} % XXX hack to align to top, see https://tex.stackexchange.com/a/11632
        \raggedleft #1
    \end{minipage}% XXX necessary comment to avoid unwanted space
    \hspace{\cvcolumngapwidth}% XXX necessary comment to avoid unwanted space
    \begin{minipage}[t]{\cvrightcolumnwidth}
        \vspace{0mm} % XXX hack to align to top, see https://tex.stackexchange.com/a/11632
        #2
    \end{minipage}

    % space after
    \vspace{\cvafteritemskipamount}
}

% typesets a name, with appropriate vertical space after
% @param #1 name text
\newcommand{\cvname}[1]{
    % name
    \cvnamestyle{#1}

    % space after
    \vspace{\cvafternameskipamount}
}

% typesets a line of personal info beginning with an icon, with appropriate vertical space after
% @param #1 parameters for the \includegraphics command used to include the icon
% @param #2 icon filename
% @param #3 line text
\newcommand{\cvpersonalinfolinewithicon}[3]{
    % icon, vertically aligned with text (see https://tex.stackexchange.com/a/129463)
    \raisebox{.5\fontcharht\font`E-.5\height}{\includegraphics[#1]{#2}}
    % text
    #3

    % space after
    \vspace{\cvafterpersonalinfolineskipamount}
}

% creates a "section" CV item with the given left column content, a horizontal rule in the right column, and with
% appropriate vertical space after
% @param #1 left column content (should be the section name)
\newcommand{\cvsection}[1]{
    % left and right column
    \begin{minipage}[t]{\cvleftcolumnwidth}
        \raggedleft\cvsectionstyle{#1}
    \end{minipage}% XXX necessary comment to avoid unwanted space
    \hspace{\cvcolumngapwidth}% XXX necessary comment to avoid unwanted space
    \begin{minipage}[t]{\cvrightcolumnwidth}
        \textcolor{cvrulecolor}{\rule{\cvrightcolumnwidth}{0.3mm}}
    \end{minipage}

    % space after
    \vspace{\cvaftersectionskipamount}
}

% creates a standard, multi-purpose CV item with the given left and right column contents, parskip set to cvparskip
% in the right column, and with appropriate vertical space after
% @param #1 left column content
% @param #2 right column content
\newcommand{\cvitem}[2]{
    % left and right column
    \begin{minipage}[t]{\cvleftcolumnwidth}
        \raggedleft #1
    \end{minipage}% XXX necessary comment to avoid unwanted space
    \hspace{\cvcolumngapwidth}% XXX necessary comment to avoid unwanted space
    \begin{minipage}[t]{\cvrightcolumnwidth}
        \setlength{\parskip}{\cvparskip} #2
    \end{minipage}

    % space after
    \vspace{\cvafteritemskipamount}
}

% typesets a title, with appropriate vertical space after
% @param #1 title text
\newcommand{\cvtitle}[1]{
    % title
    \cvtitlestyle{#1}

    % space after
    \vspace{\cvaftertitleskipamount}
    % XXX need to subtract cvparskip here, because it is automatically inserted after the title "paragraph"
    \vspace{-\cvparskip}
}


% header and footer
% -----------------

% set empty header and footer
\pagestyle{empty}



% preamble end/document start
% ===========================

\begin{document}


% personal info
% -------------

\cvpersonalinfo{
    % photo
    \includegraphics[height=36mm]{photo.jpg}
}{
    % name
    \cvname{Teich, Juan Ignacio}

    % address
    \cvpersonalinfolinewithicon{height=4mm}{072-location.pdf}{
        Azcuénaga 2529, Martinez, Buenos Aires, Argentina
    }

    % phone number
    \cvpersonalinfolinewithicon{height=4mm}{067-phone.pdf}{
        +54 11\,6491\,6389
    }

    % email address
    \cvpersonalinfolinewithicon{height=4mm}{070-envelop.pdf}{
        juanignacioteich@gmail.com
    }

    % % LinkedIn account
    % \cvpersonalinfolinewithicon{height=4mm}{458-linkedin.pdf}{
    %     fake-name-123456789
    % }

    % date of birth
        22-03-1999  -   24 años
}


% work experience
% ---------------
\cvsection{OBJETIVOS PROFESIONALES}

\vspace{\cvbetweensectionandheadingextraskipamount}
\cvitem{\cvheadingstyle{}}{
    Continuar con mi carrera profesional, en busca de proyectos desafiantes de Ingeniería Mecánica, poniendo en uso las herramientas ganadas durante la carrera.

    Busco orientarme a la simulación para la solución de problemas de la industria.
}


\cvsection{EXPERIENCIA LABORAL}

\cvitem{
	\cvdurationstyle{Octubre 2023 -- Presente}
}{
	\begin{minipage}{0.1\textwidth}
		\centering
		\includegraphics[width=0.8\textwidth]{Logo_STAMM.png}   
	\end{minipage}
	\cvtitle{Numerical Simulations Specialist}
	
	\textbf{STÄMM Biotech}
	
	
}

\cvitem{
    \cvdurationstyle{Noviembre 2022 -- Octubre 2023}
}{
    \begin{minipage}{0.1\textwidth}
        \centering
        \includegraphics[width=0.8\textwidth]{Logo_SG.png}   
    \end{minipage}
    \cvtitle{Líder Junior de Proyectos Estratégicos}

    \textbf{Saint-Gobain Argentina}
	
	Tareas desarrolladas:
    \begin{itemize}
        \item Líder de proyectos estratégicos, incluyendo el análisis de las necesidades de la companía para la continuidad del negocio, el diseño básico del proyecto y la implementación del mismo
        \item Coordinación de equipos multidisciplinarios y stakeholders, buscando soluciones consensuadas cumpliendo alcance, tiempo y costo disponible
        \item Soporte y análisis técnico para adquisiciones de nuevos negocios a nivel regional
    \end{itemize}
	
	Habilidades y herramientas:
	\begin{itemize}
		\item AutoCAD, SolidWorks, MS Excel, MS PowerPoint, MS Planner, MS Project.
	\end{itemize}

}

% Fake Company 2
\cvitem{
    \cvdurationstyle{Marzo 2020 -- Marzo 2021}
}{
    \begin{minipage}{0.1\textwidth}
        \centering
        \includegraphics[width=0.8\textwidth]{Logo_FIUBA.png}   
    \end{minipage}
    \cvtitle{Ayudante de Cátedra de Termodinámica I}

    Facultad de Ingeniería, Universidad de Buenos Aires
}

% Fake Company 1
\cvitem{
    \cvdurationstyle{Enero 2017 -- Julio 2018}
}{
    \cvtitle{Profesor Particular de alumnos de Secundario}

    \begin{itemize}[leftmargin=*]
        \item Ayuda para planificación y explicación de asignaturas para alumnos de secundario
    \end{itemize}
}


% education
% ---------

\cvsection{EDUCACIÓN}

% master's
\cvitem{
    \cvdurationstyle{2017 -- Presente}
}{
    \begin{minipage}{0.1\textwidth}
        \centering
        \includegraphics[width=0.8\textwidth]{Logo_FIUBA.png}   
    \end{minipage}      
    \cvtitle{Ingeniería Mecánica - en curso}

    Facultad de Ingeniería, Universidad de Buenos Aires
    
    \begin{itemize}[leftmargin=*]
        \item Materias y finales finalizados.
        \item Tesis en curso - \textit{Análisis paramétrico de modelos de actuador discal para simulación de aerogeneradores en parques Eólicos} - CSC.
        \item Fecha de finalización de tesis: agosto de 2024.
    \end{itemize}
}

%curso
\cvitem{
    \cvdurationstyle{Sep. 2021 -- Nov. 2021}
}{
    \begin{minipage}{0.1\textwidth}
        \centering
        \includegraphics[width=0.8\textwidth]{Logo_edx.png}   
    \end{minipage}      
    \cvtitle{A Hands-on Introduction to Engineering Simulations}
    edX \& Cornell University

    \begin{itemize}
        \item Curso de ANSYS
    \end{itemize}
}

% bachelor's
\cvitem{
    \cvdurationstyle{2011 -- 2016}
}{
    \begin{minipage}{0.1\textwidth}
        \centering
        \includegraphics[width=0.8\textwidth]{Logo_HC.png}   
    \end{minipage}  
    \cvtitle{Bachiller Bilingüe en Ciencias Sociales}

    Colegio Holy Cross
}


% skills
% ------
\newpage
\cvsection{HABILIDADES}

\vspace{\cvbetweensectionandheadingextraskipamount}

% languages
\cvitem{
    \cvheadingstyle{Idiomas}
}{
    Inglés - Bilingüe
    \begin{itemize}
        \item Certificados IGCSE, Universidad de Cambridge
    \end{itemize}

    Francés - Básico
    \begin{itemize}
        \item Certificados A1 y A2
    \end{itemize}
}

%software
\cvitem{
    \cvheadingstyle{Software}
}{
    Fusion 360 - Diseño de piezas mecánicas 3D
    
    Paquete Office (Excel, Word, PowerPoint, Project)

    AutoCAD
    
    ANSYS - Simulación para ingeniería
    
    LaTex - Elaboración de documentos técnicos y científicos
    
    OpenFoam - Simulación de fluidos

    MatLab

    SolidWorks

    Linux
}


%\cvsection{INTERESES PERSONALES}

%\vspace{\cvbetweensectionandheadingextraskipamount}
%\cvitem{\cvheadingstyle{}}{
%    Tengo interés por los deportes en general, en partícular realizo rutinas de musculación en un gimnasio y he realizado múltiples deportes de equipo e individuales en el pasado.
%
%    Soy un aficionado de la lectura, en particular de fantasía, ciencia-ficción, crimen y suspenso.
%}

% fake skills
% \cvitem{
%     \cvheadingstyle{Fake skills}
% }{
%     Fake skill 1
%     \begin{itemize}
%         \item fake skill description (excepteur sint occaecat cupidatat non proident)
%         \item fake skill details (sunt in culpa qui officia deserunt mollit anim id est laborum)
%     \end{itemize}

%     Fake skill 2

%     Fake skill 3
% }

% % completely fake skills
% \cvitem{
%     \cvheadingstyle{Completely fake skills}
% }{
%     Completely fake skill 1
%     \begin{itemize}
%         \item completely fake skill description
%     \end{itemize}

%     Completely fake skill 2
% }


% % additional info
% % ---------------

% \cvsection{ADDITIONAL INFORMATION}

% \vspace{\cvbetweensectionandheadingextraskipamount}

% % driving licence
% \cvitem{
%     \cvheadingstyle{Driving licence}
% }{
%     Fake category
% }

% % interests
% \cvitem{
%     \cvheadingstyle{Interests}
% }{
%     Fake interest 1, fake interest 2, fake interest 3
% }

\end{document}